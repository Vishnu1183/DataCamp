######################### Advanced Applications of Simulation  ########################

##### Modeling Corn Production
# Initialize variables
cost = 5000
rain = np.random.normal(50, 15)

# Corn Production Model
def corn_produced(rain, cost):
  mean_corn = 100*(cost**0.1)*(rain**0.2)
  corn = np.random.poisson(mean_corn)
  return corn

# Simulate and print corn production
corn_result = corn_produced(rain, cost)
print("Simulated Corn Production = {}".format(corn_result))

##### Modeling Profits
# Function to calculate profits
def profits(cost):
    rain = np.random.normal(50, 15)
    price = np.random.normal(40, 10)
    supply = corn_produced(rain, cost)
    demand = corn_demanded(price)
    equil_short = supply <= demand
    if equil_short == True:
        tmp = supply*price - cost
        return tmp
    else:
        tmp2 = demand*price - cost
        return tmp2
result = profits(cost)
print("Simulated profit = {}".format(result))

##### Optimizing Costs
# Initialize results and cost_levels variables
sims, results = 1000, {}
cost_levels = np.arange(100, 5100, 100)

# For each cost level, simulate profits and store mean profit
for cost in cost_levels:
    tmp_profits = []
    for i in range(sims):
        tmp_profits.append(profits(cost))
    results[cost] = np.mean(tmp_profits)
    
# Get the cost that maximizes average profit
cost_max = [x for x in results.keys() if results[x] == max(results.values())][0]
print("Average profit is maximized when cost = {}".format(cost_max))

##### Integrating a Simple Function
# Define the sim_integrate function
def sim_integrate(func, xmin, xmax, sims):
    x = np.random.uniform(xmin, xmax, sims)
    y = np.random.uniform(min(min(func(x)), 0), max(func(x)), sims)
    area = (max(y) - min(y))*(xmax-xmin)
    result = area * sum(abs(y) < abs(func(x)))/sims
    return result

# Call the sim_integrate function and print results
result = sim_integrate(func = lambda x: x*np.exp(x), xmin = 0, xmax = 1, sims = 50)
print("Simulated answer = {}, Actual Answer = 1".format(result))

###### Calculating the value of pi
# Initialize sims and circle_points
sims, circle_points = 10000, 0 

for i in range(sims):
    # Generate the two coordinates of a point
    point = np.random.uniform(-1, 1, 2)
    # if the point lies within the unit circle, increment counter
    within_circle = point[0]**2 + point[1]**2 <= 1
    if within_circle == True:
        circle_points +=1
        
# Estimate pi as 4 times the avg number of points in the circle.
pi_sim = 4*circle_points/sims
print("Simulated value of pi = {}".format(pi_sim))

##### Factors influencing Statistical Power
Number of Simulations

##### Power Analysis - Part I
# Initialize effect_size, sample_size, control_mean, control_sd
effect_size, sample_size, control_mean, control_sd = 0.05, 50, 1, 0.5

# Simulate control_time_spent and treatment_time_spent, assuming equal variance
control_time_spent = np.random.normal(loc=control_mean, scale=control_sd, size=sample_size)
treatment_time_spent = np.random.normal(loc=control_mean*(1+effect_size), scale=control_sd, size=sample_size)

# Run the t-test and get the p_value
t_stat, p_value = st.ttest_ind(treatment_time_spent, control_time_spent)
stat_sig = p_value < 0.05
print("P-value: {}, Statistically Significant? {}".format(p_value, stat_sig))

##### Power Analysis - Part II
sample_size = 50

# Keep incrementing sample size by 10 till we reach required power
while 1:
    control_time_spent = np.random.normal(loc=control_mean, scale=control_sd, size=(sample_size, sims))
    treatment_time_spent = np.random.normal(loc=control_mean*(1+effect_size), scale=control_sd, size=(sample_size, sims))
    t, p = st.ttest_ind(treatment_time_spent, control_time_spent)
    
    # Power is the fraction of times in the simulation when the p-value was less than 0.05
    power = (p < 0.05).sum()/sims
    if power >= 0.8: 
        break
    else: 
        sample_size += 10
print("For 80% power, sample size required = {}".format(sample_size))

##### Portfolio Simulation - Part I
# rates is a Normal random variable and has size equal to number of years
def portfolio_return(yrs, avg_return, sd_of_return, principal):
    np.random.seed(123)
    rates = np.random.normal(loc=avg_return, scale=sd_of_return, size=yrs)
    # Calculate the return at the end of the period
    end_return = principal
    for x in rates:
        end_return = end_return*(1+x)
    return end_return

result = portfolio_return(yrs = 5, avg_return = 0.07, sd_of_return = 0.15, principal = 1000)
print("Portfolio return after 5 years = {}".format(result))

##### Portfolio Simulation - Part II
# Run 1,000 iterations and store the results
sims, rets = 1000, []

for i in range(sims):
    rets.append(portfolio_return(yrs = 10, avg_return = 0.07, 
                                 volatility = 0.3, principal = 10000))

# Calculate the 95% CI
lower_ci = np.percentile(rets, 2.5) 
upper_ci = np.percentile(rets, 97.5)
print("95% CI of Returns: Lower = {}, Upper = {}".format(lower_ci, upper_ci))

##### Portfolio Simulation - Part III
for i in range(sims):
    rets_stock.append(portfolio_return(yrs = 10, avg_return = 0.07, volatility = 0.3, principal = 10000))
    rets_bond.append(portfolio_return(yrs = 10, avg_return = 0.04, volatility = 0.1, principal = 10000))

# Calculate the 25th percentile of the distributions and the amount you'd lose or gain
rets_stock_perc = np.percentile(rets_stock, 25)
rets_bond_perc = np.percentile(rets_bond, 25)
additional_returns = rets_stock_perc - rets_bond_perc
print("Sticking to stocks gets you an additional return of {}".format(additional_returns))

