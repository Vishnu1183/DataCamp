############### Probability & data generation process  ############

##### Queen and spade
16/52

##### Two of a kind
# Shuffle deck & count card occurrences in the hand
n_sims, two_kind = 10000, 0
for i in range(n_sims):
    np.random.shuffle(deck_of_cards)
    hand, cards_in_hand = deck_of_cards[0:5], {}
    for card in hand:
        # Use .get() method on cards_in_hand
        cards_in_hand[card[1]] = cards_in_hand.get(card[1], 0) + 1
    
    # Condition for getting at least 2 of a kind
    highest_card = max(cards_in_hand.values())
    if  highest_card>=2: 
        two_kind += 1

print("Probability of seeing at least two of a kind = {} ".format(two_kind/n_sims))

##### Game of thirteen
# Pre-set constant variables
deck, sims, coincidences = np.arange(1, 14), 10000, 0

for _ in range(sims):
    # Draw all the cards without replacement to simulate one game
    draw = np.random.choice(deck, size=13, replace=False)
    # Check if there are any coincidences
    coincidence = (draw == list(np.arange(1, 14))).any()
    if coincidence == True: 
        coincidences += 1

# Calculate probability of winning
prob_of_winning = 1- coincidences/sims
print("Probability of winning = {}".format(prob_of_winning))

##### The conditional urn
# Initialize success, sims and urn
success, sims = 0, 5000
urn = ['w']*7 + ['b']*6

for _ in range(sims):
    # Draw 4 balls without replacement
    draw = np.random.choice(urn, replace=False, size=4)
    # Count the number of successes
    if (draw[0] == 'w') & (draw[1] == 'b') & (draw[2] == 'w') & (draw[3] == 'b'):
        success +=1

print("Probability of success = {}".format(success/sims))

##### Birthday problem 1
# Draw a sample of birthdays & check if each birthday is unique
days = np.arange(1,366)
people = 2

def birthday_sim(people):
    sims, unique_birthdays = 2000, 0 
    for _ in range(sims):
        draw = np.random.choice(days, size=people, replace=True)
        if len(draw) == len(set(draw)): 
            unique_birthdays += 1
    out = 1 - unique_birthdays / sims
    return out

##### Birthday problem 2
# Break out of the loop if probability greater than 0.5
while (people > 0):
    prop_bds = birthday_sim(people)
    if prop_bds > 0.5: 
        break
    people += 1

print("With {} people, there's a 50% chance that two share a birthday.".format(people))

##### Full house
#Shuffle deck & count card occurrences in the hand
n_sims, full_house, deck_of_cards = 50000, 0, deck.copy() 
for i in range(n_sims):
    np.random.shuffle(deck_of_cards)
    hand, cards_in_hand = deck_of_cards[0:5], {}
    for card in hand:
        # Use .get() method to count occurrences of each card
        cards_in_hand[card[1]] = cards_in_hand.get(card[1], 0) + 1
        
    # Condition for getting full house
    condition = (max(cards_in_hand.values()) ==3) & (min(cards_in_hand.values())==2)
    if  condition == True: 
        full_house += 1
print("Probability of seeing a full house = {}".format(full_house/n_sims))

##### Driving test 1
sims, outcomes, p_rain, p_pass = 1000, [], 0.40, {'sun':0.9, 'rain':0.3}

def test_outcome(p_rain):
    # Simulate whether it will rain or not
    weather = np.random.choice(['rain', 'sun'], p=[p_rain, 1-p_rain])
    # Simulate and return whether you will pass or fail
    return np.random.choice(['pass', 'fail'], p=[p_pass[weather], 1-p_pass[weather]])

##### Driving test 2
for _ in range(sims):
    outcomes.append(test_outcome(p_rain))

# Calculate fraction of outcomes where you pass
pass_outcomes_frac = sum([x == 'pass' for x in outcomes])/sims
print("Probability of Passing the driving test = {}".format(pass_outcomes_frac))

##### National elections
outcomes, sims, probs = [], 1000, p

for _ in range(sims):
    # Simulate elections in the 50 states
    election = np.random.binomial(p=probs, n=1)
    # Get average of Red wins and add to `outcomes`
    outcomes.append(election.mean())

# Calculate probability of Red winning in less than 45% of the states
prob_red_wins = sum([(x < 0.45) for x in outcomes])/len(outcomes)
print("Probability of Red winning in less than 45% of the states = {}".format(prob_red_wins))

##### Fitness goals
# Simulate steps & choose prob 
for _ in range(sims):
    w = []
    for i in range(days):
        lam = np.random.choice([5000, 15000], p=[0.6, 0.4], size=1)
        steps = np.random.poisson(lam)
        if steps > 10000: 
            prob = [0.2, 0.8]
        elif steps < 8000: 
            prob = [0.8, 0.2]
        else:
            prob = [0.5, 0.5]
        w.append(np.random.choice([1, -1], p=prob))
    outcomes.append(sum(w))

# Calculate fraction of outcomes where there was a weight loss
weight_loss_outcomes_frac = sum([x < 0 for x in outcomes])/len(outcomes)
print("Probability of Weight Loss = {}".format(weight_loss_outcomes_frac))

##### Sign up Flow
# Initialize click-through rate and signup rate dictionaries
ct_rate = {'low':0.01, 'high':np.random.uniform(low=0.01, high=1.2*0.01)}
su_rate = {'low':0.2, 'high':np.random.uniform(low=0.2, high=1.2*0.2)}

def get_signups(cost, ct_rate, su_rate, sims):
    lam = np.random.normal(loc=100000, scale=2000, size=sims)
    # Simulate impressions(poisson), clicks(binomial) and signups(binomial)
    impressions = np.random.poisson(lam=lam)
    clicks = np.random.binomial(impressions, p=ct_rate[cost])
    signups = np.random.binomial(clicks, p=su_rate[cost])
    return signups

print("Simulated Signups = {}".format(get_signups('high', ct_rate, su_rate, 1)))

##### Purchase Flow
def get_revenue(signups):
    rev = []
    np.random.seed(123)
    for s in signups:
        # Model purchases as binomial, purchase_values as exponential
        purchases = np.random.binomial(s, p=0.1)
        purchase_values = np.random.exponential(scale=1000, size=purchases)
        
        # Append to revenue the sum of all purchase values.
        rev.append(purchase_values.sum())
    return rev

print("Simulated Revenue = ${}".format(get_revenue(get_signups('low', ct_rate, su_rate, 1))[0]))

##### Probability of losing money
# Initialize sims
sims, cost_diff = 10000, 3000

# Get revenue when the cost is 'low' and when the cost is 'high'
rev_low = get_revenue(get_signups('low', ct_rate, su_rate, sims))
rev_high = get_revenue(get_signups('high', ct_rate, su_rate, sims))

# calculate fraction of times rev_high - rev_low is less than cost_diff
frac = sum([rev_high[i] - rev_low[i] < cost_diff for i in range(len(rev_low))])/len(rev_low)
print("Probability of losing money = {}".format(frac))
