########################## The R Package Structure  ####################

$$$$$$  The structure of an R package
# Use the create function to set up your first package
create("datasummary")

# Take a look at the files and folders in your package
dir("datasummary")


$$$$$$ Contents of an R package
R


$$$$$$ Writing a simple function
# Create numeric_summary() function
numeric_summary <- function(x, na.rm) {

    # Include an error if x is not numeric
    if(!is.numeric(x)){
        stop("Data must be numeric")
    }
    
    # Create data frame
    data.frame( min = min(x, na.rm = na.rm),
                median = median(x, na.rm = na.rm),
                sd = sd(x, na.rm = na.rm),
                max = max(x, na.rm = na.rm))
}

# Test numeric_summary() function
numeric_summary(airquality$Ozone, na.rm=T)



$$$$$$ Including functions in a package
# What is in the R directory before adding a function?
dir("datasummary/R")

# Use the dump() function to write the numeric_summary function
dump("numeric_summary", file = "datasummary/R/numeric_summary.R")

# Verify that the file is in the correct directory
dir("datasummary/R")


$$$$$$ Package names
tidydata



$$$$$$ Writing a DESCRIPTION file
Functions that the package contains


$$$$$$ Detailing authors, maintainers and contributors
 cre


$$$$$$ The use_* functions
# What is in the package at the moment?
dir('datasummary')

# Add the weather data
use_data(weather, pkg = "datasummary", overwrite = TRUE)

# Add a vignette called "Generating Summaries with Data Summary"
use_vignette("Generating_Summaries_with_Data_Summary", pkg = "datasummary")

# What directories do you now have in your package now?
dir("datasummary")


$$$$$$ Best practice for structuring code
data_summary <- function(x, na.rm = TRUE){
  
  num_data <- select_if(x, .predicate = is.numeric) 
  
  map_df(num_data, .f = numeric_summary, na.rm = TRUE, .id = "ID")
  
}

# Write the function to the R directory
dump("data_summary", file = "datasummary/R/data_summary.R")

