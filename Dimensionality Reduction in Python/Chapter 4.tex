################### Feature extraction #####################

$$$$$ Manual feature extraction I
# Calculate the price from the quantity sold and revenue
sales_df['price'] = sales_df['revenue']/sales_df['quantity']

# Drop the quantity and revenue features
reduced_df = sales_df.drop(['quantity', 'revenue'], axis=1)

print(reduced_df.head())


$$$$$ Manual feature extraction II
# Calculate the mean height
height_df['height'] = height_df[['height_1','height_2','height_3']].mean(axis=1)

# Drop the 3 original height features
reduced_df = height_df.drop(['height_1','height_2','height_3'], axis=1)

print(reduced_df.head())


$$$$$ Principal component intuition 
People with a negative component for the yellow vector have long forearms relative to their upper arms.


$$$$$ Calculating Principal Components
# Create a pairplot to inspect ansur_df
sns.pairplot(ansur_df)

plt.show()


$$$$$ Calculating Principal Components
from sklearn.preprocessing import StandardScaler

# Create the scaler and standardize the data
scaler = StandardScaler()
ansur_std = scaler.fit_transform(ansur_df)


$$$$$ Calculating Principal Components
from sklearn.preprocessing import StandardScaler
from sklearn.decomposition import PCA

# Create the scaler and standardize the data
scaler = StandardScaler()
ansur_std = scaler.fit_transform(ansur_df)

# Create the PCA instance and fit and transform the data with pca
pca = PCA()
pc = pca.fit_transform(ansur_std)

# This changes the numpy array output back to a dataframe
pc_df = pd.DataFrame(pc, columns=['PC 1', 'PC 2', 'PC 3', 'PC 4'])


$$$$$ Calculating Principal Components
from sklearn.preprocessing import StandardScaler
from sklearn.decomposition import PCA

# Create the scaler
scaler = StandardScaler()
ansur_std = scaler.fit_transform(ansur_df)

# Create the PCA instance and fit and transform the data with pca
pca = PCA()
pc = pca.fit_transform(ansur_std)
pc_df = pd.DataFrame(pc, columns=['PC 1', 'PC 2', 'PC 3', 'PC 4'])

# Create a pairplot of the principal component dataframe
sns.pairplot(pc_df)
plt.show()


$$$$$ PCA on a larger dataset
from sklearn.preprocessing import StandardScaler
from sklearn.decomposition import PCA

# Scale the data
scaler = StandardScaler()
ansur_std = scaler.fit_transform(ansur_df)

# Apply PCA
pca = PCA()
pca.fit(ansur_std)


$$$$$ PCA explained variance
# Inspect the explained variance ratio per component
print(pca.explained_variance_ratio_)



$$$$$ PCA explained variance
About 3.77%


$$$$$ PCA explained variance
# Print the cumulative sum of the explained variance ratio
print(pca.explained_variance_ratio_.cumsum())


$$$$$ PCA explained variance
4 principal components


$$$$$ Understanding the components
# Build the pipeline
pipe = Pipeline([('scaler', StandardScaler()),
        		 ('reducer', PCA(n_components=2))])


$$$$$  Understanding the components
# Build the pipeline
pipe = Pipeline([('scaler', StandardScaler()),
        		 ('reducer', PCA(n_components=2))])

# Fit it to the dataset and extract the component vectors
pipe.fit(poke_df)
vectors = pipe.steps[1][1].components_.round(2)

# Print feature effects
print('PC 1 effects = ' + str(dict(zip(poke_df.columns, vectors[0]))))
print('PC 2 effects = ' + str(dict(zip(poke_df.columns, vectors[1]))))


$$$$$  Understanding the components
All features have a similar positive effect. PC 1 can be interpreted as a measure of overall quality (high stats).


$$$$$  Understanding the components
Defense has a strong positive effect on the second component and speed a strong negative one. This component quantifies an agility vs. armor & protection trade-off.


$$$$$ PCA for feature exploration
# Build the pipeline
pipe = Pipeline([('scaler', StandardScaler()),
                 ('reducer', PCA(n_components=2))])

# Fit the pipeline to poke_df and transform the data
pc = pipe.fit_transform(poke_df)

print(pc)


$$$$$ PCA for feature exploration
pipe = Pipeline([('scaler', StandardScaler()),
                 ('reducer', PCA(n_components=2))])

# Fit the pipeline to poke_df and transform the data
pc = pipe.fit_transform(poke_df)

# Add the 2 components to poke_cat_df
poke_cat_df['PC 1'] = pc[:,0]
poke_cat_df['PC 2'] = pc[:,1]

$$$$$ PCA for feature exploration
pipe = Pipeline([('scaler', StandardScaler()),
                 ('reducer', PCA(n_components=2))])

# Fit the pipeline to poke_df and transform the data
pc = pipe.fit_transform(poke_df)

# Add the 2 components to poke_cat_df
poke_cat_df['PC 1'] = pc[:, 0]
poke_cat_df['PC 2'] = pc[:, 1]

# Use the Type feature to color the PC 1 vs PC 2 scatterplot
sns.scatterplot(data=poke_cat_df, 
                x='PC 1', y='PC 2', hue='Type')
plt.show()


$$$$$ PCA for feature exploration
pipe = Pipeline([('scaler', StandardScaler()),
                 ('reducer', PCA(n_components=2))])

# Fit the pipeline to poke_df and transform the data
pc = pipe.fit_transform(poke_df)

# Add the 2 components to poke_cat_df
poke_cat_df['PC 1'] = pc[:, 0]
poke_cat_df['PC 2'] = pc[:, 1]

# Use the Legendary feature to color the PC 1 vs PC 2 scatterplot
sns.scatterplot(data=poke_cat_df, 
                x='PC 1', y='PC 2', hue='Legendary')
plt.show()


$$$$$ PCA in a model pipeline
# Build the pipeline
pipe = Pipeline([
        ('scaler', StandardScaler()),
        ('reducer', PCA(2)),
        ('classifier', RandomForestClassifier(random_state=0))])
        

$$$$$ PCA in a model pipeline
# Build the pipeline
pipe = Pipeline([
        ('scaler', StandardScaler()),
        ('reducer', PCA(n_components=2)),
        ('classifier', RandomForestClassifier(random_state=0))])

# Fit the pipeline to the training data
pipe.fit(X_train,y_train)

# Prints the explained variance ratio
print(pipe.steps[1][1].explained_variance_ratio_)


$$$$$ PCA in a model pipeline
# Build the pipeline
pipe = Pipeline([
        ('scaler', StandardScaler()),
        ('reducer', PCA(n_components=2)),
        ('classifier', RandomForestClassifier(random_state=0))])

# Fit the pipeline to the training data
pipe.fit(X_train, y_train)

# Score the accuracy on the test set
accuracy = pipe.score(X_test, y_test)

# Prints the model accuracy
print('{0:.1%} test set accuracy'.format(accuracy))


$$$$$ PCA in a model pipeline
# Build the pipeline
pipe = Pipeline([
        ('scaler', StandardScaler()),
        ('reducer', PCA(n_components=3)),
        ('classifier', RandomForestClassifier(random_state=0))])

# Fit the pipeline to the training data
pipe.fit(X_train, y_train)

# Score the accuracy on the test set
accuracy = pipe.score(X_test, y_test)

# Prints the explained variance ratio and accuracy
print(pipe.steps[1][1].explained_variance_ratio_)
print('{0:.1%} test set accuracy'.format(accuracy))



$$$$$ Selecting the proportion of variance to keep
# Pipe a scaler to PCA selecting 80% of the variance
pipe = Pipeline([('scaler', StandardScaler()),
        		 ('reducer', PCA(0.8))])


$$$$$ Selecting the proportion of variance to keep
# Pipe a scaler to PCA selecting 80% of the variance
pipe = Pipeline([('scaler', StandardScaler()),
        		 ('reducer', PCA(n_components=0.8))])

# Fit the pipe to the data
pipe.fit(ansur_df)

print('{} components selected'.format(len(pipe.steps[1][1].components_)))


$$$$$ Selecting the proportion of variance to keep
# Let PCA select 90% of the variance
pipe = Pipeline([('scaler', StandardScaler()),
        		 ('reducer', PCA(n_components=0.9))])

# Fit the pipe to the data
pipe.fit(ansur_df)

print('{} components selected'.format(len(pipe.steps[1][1].components_)))


$$$$$ Selecting the proportion of variance to keep
12



$$$$$ Choosing the number of components
# Pipeline a scaler and PCA selecting 10 components
pipe = Pipeline([('scaler', StandardScaler()),
        		 ('reducer', PCA(10))])


$$$$$ Choosing the number of components
# Pipeline a scaler and pca selecting 10 components
pipe = Pipeline([('scaler', StandardScaler()),
        		 ('reducer', PCA(n_components=10))])

# Fit the pipe to the data
pipe.fit(ansur_df)


$$$$$ Choosing the number of components
# Pipeline a scaler and pca selecting 10 components
pipe = Pipeline([('scaler', StandardScaler()),
        		 ('reducer', PCA(n_components=10))])

# Fit the pipe to the data
pipe.fit(ansur_df)

# Plot the explained variance ratio
plt.plot(pipe.steps[1][1].explained_variance_ratio_)

plt.xlabel('Principal component index')
plt.ylabel('Explained variance ratio')
plt.show()


$$$$$ Choosing the number of components
3


$$$$$ PCA for image compression
# Plot the MNIST sample data
plot_digits(X_test)


$$$$$ PCA for image compression
# Transform the input data to principal components
pc = pipe.transform(X_test)

# Prints the number of features per dataset
print("X_test has {} features".format(X_test.shape[1]))
print("pc has {} features".format(pc.shape[1]))


$$$$$ PCA for image compression
# Transform the input data to principal components
pc = pipe.transform(X_test)

# Inverse transform the components to original feature space
X_rebuilt = pipe.inverse_transform(pc)

# Prints the number of features
print("X_rebuilt has {} features".format(X_rebuilt.shape[1]))


$$$$$ PCA for image compression
# Transform the input data to principal components
pc = pipe.transform(X_test)

# Inverse transform the components to original feature space
X_rebuilt = pipe.inverse_transform(pc)

# Plot the reconstructed data
plot_digits(X_rebuilt)



