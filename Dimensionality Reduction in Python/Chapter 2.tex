############################## Feature selection I, selecting for feature information ###################

$$$$$ Train - test split
# Import train_test_split()
from sklearn.model_selection import train_test_split

# Select the Gender column as the feature to be predicted (y)
y = ansur_df['Gender']

# Remove the Gender column to create the training data
X = ansur_df.drop('Gender', axis = 1)

# Perform a 70% train and 30% test data split
X_train, X_test, y_train, y_test = train_test_split(X, y, test_size=0.3)

print("{} rows in test set vs. {} in training set. {} Features.".format(X_test.shape[0], X_train.shape[0], X_test.shape[1]))


$$$$$ Fitting and testing the model
# Import SVC from sklearn.svm and accuracy_score from sklearn.metrics
from sklearn.svm import SVC
from sklearn.metrics import accuracy_score

# Create an instance of the Support Vector Classification class
svc = SVC()

# Fit the model to the training data
svc.fit(X_train, y_train)

# Calculate accuracy scores on both train and test data
accuracy_train = accuracy_score(y_train, svc.predict(X_train))
accuracy_test = accuracy_score(y_test, svc.predict(X_test))

print("{0:.1%} accuracy on test set vs. {1:.1%} on training set".format(accuracy_test, accuracy_train))


$$$$$ Accuracy after dimensionality reduction
# Assign just the 'neckcircumferencebase' column from ansur_df to X
X = ansur_df[['neckcircumferencebase']]

# Split the data, instantiate a classifier and fit the data
X_train, X_test, y_train, y_test = train_test_split(X, y, test_size=0.3)
svc = SVC()
svc.fit(X_train, y_train)

# Calculate accuracy scores on both train and test data
accuracy_train = accuracy_score(y_train, svc.predict(X_train))
accuracy_test = accuracy_score(y_test, svc.predict(X_test))

print("{0:.1%} accuracy on test set vs. {1:.1%} on training set".format(accuracy_test, accuracy_train))


$$$$$ Finding a good variance threshold
# Create the boxplot
head_df.boxplot()

plt.show()


$$$$$ Finding a good variance threshold
# Normalize the data
normalized_df = head_df / head_df.mean()

normalized_df.boxplot()
plt.show()


$$$$$ Finding a good variance threshold
# Normalize the data
normalized_df = head_df / head_df.mean()

# Print the variances of the normalized data
print(normalized_df.var())


$$$$$ Finding a good variance threshold
1.0e-03


$$$$$ Features with low variance
from sklearn.feature_selection import VarianceThreshold

# Create a VarianceThreshold feature selector
sel = VarianceThreshold(threshold=0.001)

# Fit the selector to normalized head_df
sel.fit(head_df / head_df.mean())

# Create a boolean mask
mask = sel.get_support()

# Apply the mask to create a reduced dataframe
reduced_df = head_df.loc[:, mask]

print("Dimensionality reduced from {} to {}.".format(head_df.shape[1], reduced_df.shape[1]))  


$$$$$ Removing features with many missing values
Between 0.9 and 1.0.


$$$$$ Removing features with many missing values
# Create a boolean mask on whether each feature less than 50% missing values.
mask = school_df.isna().sum() / len(school_df) < 0.5

# Create a reduced dataset by applying the mask
reduced_df = school_df.drop([mask], axis = 1)

print(school_df.shape)
print(reduced_df.shape)


$$$$$ Correlation intuition
The correlation coefficient of A to B is equal to that of B to A.


$$$$$ Inspecting the correlation matrix
0.702178


$$$$$ Visualizing the correlation matrix
# Create the correlation matrix
corr = ansur_df.corr()

# Draw the heatmap
sns.heatmap(corr,  cmap=cmap, center=0, linewidths=1, annot=True, fmt=".2f")
plt.show()

$$$$$ Visualizing the correlation matrix
# Create the correlation matrix
corr = ansur_df.corr()

# Generate a mask for the upper triangle
mask = np.triu(np.ones_like(corr, dtype=bool))


$$$$$ Visualizing the correlation matrix
# Create the correlation matrix
corr = ansur_df.corr()

# Generate a mask for the upper triangle 
mask = np.triu(np.ones_like(corr, dtype=bool))

# Add the mask to the heatmap
sns.heatmap(corr, mask=mask, cmap=cmap, center=0, linewidths=1, annot=True, fmt=".2f")
plt.show()


$$$$$ Visualizing the correlation matrix
Buttock height and Crotch height.



$$$$$ Filtering out highly correlated features
# Calculate the correlation matrix and take the absolute value
corr_matrix = ansur_df.corr().abs()

# Create a True/False mask and apply it
mask = np.triu(np.ones_like(corr_matrix, dtype=bool))
tri_df = corr_matrix.mask(mask)

# List column names of highly correlated features (r > 0.95)
to_drop = [c for c in tri_df.columns if any(tri_df[c] >  0.95)]

# Drop the features in the to_drop list
reduced_df = ansur_df.drop(to_drop, axis=1)

print("The reduced dataframe has {} columns.".format(reduced_df.shape[1]))


$$$$$ Nuclear energy and pool drownings
# Print the first five lines of weird_df
print(weird_df.head())

$$$$$ Nuclear energy and pool drownings
# Put nuclear energy production on the x-axis and the number of pool drownings on the y-axis
sns.scatterplot(x='nuclear_energy', y='pool_drownings', data=weird_df)
plt.show()

$$$$$ Nuclear energy and pool drownings
# Print out the correlation matrix of weird_df
print(weird_df.corr())

$$$$$ Nuclear energy and pool drownings
Not much, correlation does not imply causation.

