############################## Feature selection II, selecting for feature information ####################

$$$$$ Building a diabetes classifier
# Fit the scaler on the training features and transform these in one go
X_train_std = scaler.fit_transform(X_train)

# Fit the logistic regression model on the scaled training data
lr.fit(X_train_std,y_train)

# Scale the test features
X_test_std = scaler.transform(X_test)

# Predict diabetes presence on the scaled test set
y_pred = lr.predict(X_test_std)

# Prints accuracy metrics and feature coefficients
print("{0:.1%} accuracy on test set.".format(accuracy_score(y_test, y_pred))) 
print(dict(zip(X.columns, abs(lr.coef_[0]).round(2))))


$$$$$ Manual Recursive Feature Elimination
# Remove the feature with the lowest model coefficient
X = diabetes_df[['pregnant', 'glucose',  'triceps', 'insulin', 'bmi', 'family', 'age']]

# Performs a 25-75% train test split
X_train, X_test, y_train, y_test = train_test_split(X, y, test_size=0.25, random_state=0)

# Scales features and fits the logistic regression model
lr.fit(scaler.fit_transform(X_train), y_train)

# Calculates the accuracy on the test set and prints coefficients
acc = accuracy_score(y_test, lr.predict(scaler.transform(X_test)))
print("{0:.1%} accuracy on test set.".format(acc)) 
print(dict(zip(X.columns, abs(lr.coef_[0]).round(2))))


$$$$$ Manual Recursive Feature Elimination
# Remove the 2 features with the lowest model coefficients
X = diabetes_df[['glucose', 'triceps', 'bmi', 'family', 'age']]

# Performs a 25-75% train test split
X_train, X_test, y_train, y_test = train_test_split(X, y, test_size=0.25, random_state=0)

# Scales features and fits the logistic regression model
lr.fit(scaler.fit_transform(X_train), y_train)

# Calculates the accuracy on the test set and prints coefficients
acc = accuracy_score(y_test, lr.predict(scaler.transform(X_test)))
print("{0:.1%} accuracy on test set.".format(acc)) 
print(dict(zip(X.columns, abs(lr.coef_[0]).round(2))))


$$$$$ Manual Recursive Feature Elimination
# Only keep the feature with the highest coefficient
X = diabetes_df[['glucose']]

# Performs a 25-75% train test split
X_train, X_test, y_train, y_test = train_test_split(X, y, test_size=0.25, random_state=0)

# Scales features and fits the logistic regression model to the data
lr.fit(scaler.fit_transform(X_train), y_train)

# Calculates the accuracy on the test set and prints coefficients
acc = accuracy_score(y_test, lr.predict(scaler.transform(X_test)))
print("{0:.1%} accuracy on test set.".format(acc)) 
print(dict(zip(X.columns, abs(lr.coef_[0]).round(2))))


$$$$$ Automatic Recursive Feature Elimination
# Create the RFE with a LogisticRegression estimator and 3 features to select
rfe = RFE(estimator=LogisticRegression(), n_features_to_select=3, verbose=1)

# Fit the eliminator to the data
rfe.fit(X_train, y_train)

# Print the features and their ranking (high = dropped early on)
print(dict(zip(X.columns, rfe.ranking_)))

# Print the features that are not eliminated
print(X.columns[rfe.support_])

# Calculates the test set accuracy
acc = accuracy_score(y_test, rfe.predict(X_test))
print("{0:.1%} accuracy on test set.".format(acc)) 


$$$$$ Building a random forest model
# Perform a 75% training and 25% test data split
X_train, X_test, y_train, y_test = train_test_split(X, y, test_size=0.25, random_state=0)

# Fit the random forest model to the training data
rf = RandomForestClassifier(random_state=0)
rf.fit(X_train, y_train)

# Calculate the accuracy
acc = accuracy_score(y_test, rf.predict(X_test))

# Print the importances per feature
print(dict(zip(X.columns, rf.feature_importances_.round(2))))

# Print accuracy
print("{0:.1%} accuracy on test set.".format(acc))



$$$$$ Random forest for feature selection
# Create a mask for features importances above the threshold
mask = rf.feature_importances_ > 0.15

# Prints out the mask
print(mask)


$$$$$ Random forest for feature selection
# Create a mask for features importances above the threshold
mask = rf.feature_importances_ > 0.15

# Apply the mask to the feature dataset X
reduced_X = X.loc[:,mask]

# prints out the selected column names
print(reduced_X.columns)


$$$$$ Recursive Feature Elimination with random forests
# Wrap the feature eliminator around the random forest model
rfe = RFE(estimator=RandomForestClassifier(), n_features_to_select=2, verbose=1)


$$$$$ Recursive Feature Elimination with random forests
# Wrap the feature eliminator around the random forest model
rfe = RFE(estimator=RandomForestClassifier(), n_features_to_select=2, verbose=1)

# Fit the model to the training data
rfe.fit(X_train,y_train)


$$$$$ Recursive Feature Elimination with random forests
# Wrap the feature eliminator around the random forest model
rfe = RFE(estimator=RandomForestClassifier(), n_features_to_select=2, verbose=1)

# Fit the model to the training data
rfe.fit(X_train, y_train)

# Create a mask using an attribute of rfe
mask = rfe.support_

# Apply the mask to the feature dataset X and print the result
reduced_X = X.loc[:, mask]
print(reduced_X.columns)


$$$$$ Recursive Feature Elimination with random forests
# Set the feature eliminator to remove 2 features on each step
rfe = RFE(estimator=RandomForestClassifier(), n_features_to_select=2, step =2, verbose=1)

# Fit the model to the training data
rfe.fit(X_train, y_train)

# Create a mask
mask = rfe.support_

# Apply the mask to the feature dataset X and print the result
reduced_X = X.loc[:, mask]
print(reduced_X.columns)



$$$$$ Creating a LASSO regressor
# Set the test size to 30% to get a 70-30% train test split
X_train, X_test, y_train, y_test = train_test_split(X, y, test_size=0.3, random_state=0)

# Fit the scaler on the training features and transform these in one go
X_train_std = scaler.fit_transform(X_train)

# Create the Lasso model
la = Lasso()

# Fit it to the standardized training data
la.fit(X_train_std, y_train)


$$$$$ Lasso model results
# Transform the test set with the pre-fitted scaler
X_test_std = scaler.transform(X_test)

# Calculate the coefficient of determination (R squared) on X_test_std
r_squared = la.score(X_test_std, y_test)
print("The model can predict {0:.1%} of the variance in the test set.".format(r_squared))

# Create a list that has True values when coefficients equal 0
zero_coef = la.coef_ == 0

# Calculate how many features have a zero coefficient
n_ignored = sum(zero_coef)
print("The model has ignored {} out of {} features.".format(n_ignored, len(la.coef_)))


$$$$$ Adjusting the regularization strength
# Find the highest alpha value with R-squared above 98%
la = Lasso(alpha = 0.1, random_state=0)

# Fits the model and calculates performance stats
la.fit(X_train_std, y_train)
r_squared = la.score(X_test_std, y_test)
n_ignored_features = sum(la.coef_ == 0)

# Print peformance stats 
print("The model can predict {0:.1%} of the variance in the test set.".format(r_squared))
print("{} out of {} features were ignored.".format(n_ignored_features, len(la.coef_)))


$$$$$ Creating a LassoCV regressor
from sklearn.linear_model import LassoCV

# Create and fit the LassoCV model on the training set
lcv = LassoCV()
lcv.fit(X_train, y_train)
print('Optimal alpha = {0:.3f}'.format(lcv.alpha_))

# Calculate R squared on the test set
r_squared = lcv.score(X_test,y_test)
print('The model explains {0:.1%} of the test set variance'.format(r_squared))

# Create a mask for coefficients not equal to zero
lcv_mask = lcv.coef_!=0
print('{} features out of {} selected'.format(sum(lcv_mask), len(lcv_mask)))


$$$$$ Ensemble models for extra votes
from sklearn.feature_selection import RFE
from sklearn.ensemble import GradientBoostingRegressor

# Select 10 features with RFE on a GradientBoostingRegressor, drop 3 features on each step
rfe_gb = RFE(estimator=GradientBoostingRegressor(), 
             n_features_to_select=10, step=3, verbose=1)
rfe_gb.fit(X_train, y_train)


$$$$$ Ensemble models for extra votes
from sklearn.feature_selection import RFE
from sklearn.ensemble import GradientBoostingRegressor

# Select 10 features with RFE on a GradientBoostingRegressor, drop 3 features on each step
rfe_gb = RFE(estimator=GradientBoostingRegressor(), 
             n_features_to_select=10, step=3, verbose=1)
rfe_gb.fit(X_train, y_train)

# Calculate the R squared on the test set
r_squared = rfe_gb.score(X_test, y_test)
print('The model can explain {0:.1%} of the variance in the test set'.format(r_squared))


$$$$$ Ensemble models for extra votes
from sklearn.feature_selection import RFE
from sklearn.ensemble import GradientBoostingRegressor

# Select 10 features with RFE on a GradientBoostingRegressor, drop 3 features on each step
rfe_gb = RFE(estimator=GradientBoostingRegressor(), 
             n_features_to_select=10, step=3, verbose=1)
rfe_gb.fit(X_train, y_train)

# Calculate the R squared on the test set
r_squared = rfe_gb.score(X_test, y_test)
print('The model can explain {0:.1%} of the variance in the test set'.format(r_squared))

# Assign the support array to gb_mask
gb_mask = rfe_gb.support_


$$$$$ Ensemble models for extra votes
from sklearn.feature_selection import RFE
from sklearn.ensemble import RandomForestRegressor

# Select 10 features with RFE on a RandomForestRegressor, drop 3 features on each step
rfe_rf = RFE(estimator=RandomForestRegressor(), 
             n_features_to_select=10, step=3, verbose=1)
rfe_rf.fit(X_train, y_train)

# Calculate the R squared on the test set
r_squared = rfe_rf.score(X_test, y_test)
print('The model can explain {0:.1%} of the variance in the test set'.format(r_squared))

# Assign the support array to gb_mask
rf_mask = rfe_rf.support_



$$$$$ Combining 3 feature selectors 
# Sum the votes of the three models
votes = np.sum([lcv_mask, rf_mask, gb_mask], axis=0)
print(votes)


$$$$$ Combining 3 feature selectors 
# Sum the votes of the three models
votes = np.sum([lcv_mask, rf_mask, gb_mask], axis=0)

# Create a mask for features selected by all 3 models
meta_mask = votes==3
print(meta_mask)


$$$$$ Combining 3 feature selectors 
# Sum the votes of the three models
votes = np.sum([lcv_mask, rf_mask, gb_mask], axis=0)

# Create a mask for features selected by all 3 models
meta_mask = votes >= 3

# Apply the dimensionality reduction on X
X_reduced = X.loc[:,meta_mask]
print(X_reduced.columns)


$$$$$ Combining 3 feature selectors 
# Sum the votes of the three models
votes = np.sum([lcv_mask, rf_mask, gb_mask], axis=0)

# Create a mask for features selected by all 3 models
meta_mask = votes >= 3

# Apply the dimensionality reduction on X
X_reduced = X.loc[:, meta_mask]

# Plug the reduced dataset into a linear regression pipeline
X_train, X_test, y_train, y_test = train_test_split(X_reduced, y, test_size=0.3, random_state=0)
lm.fit(scaler.fit_transform(X_train), y_train)
r_squared = lm.score(scaler.transform(X_test), y_test)
print('The model can explain {0:.1%} of the variance in the test set using {1:} features.'.format(r_squared, len(lm.coef_)))


$$$$$ 





