################# Exploring high dimensional data ###################

$$$$$ Finding the number of dimensions in a dataset
7 dimensions


$$$$$ Removing features without variance 1
# Remove the feature without variance from this list
number_cols = ['HP', 'Attack', 'Defense', ]
#pokemon_df.describe()


$$$$$ Removing features without variance 2
# Remove the feature without variance from this list
number_cols = ['HP', 'Attack', 'Defense']

# Leave this list as is for now
non_number_cols = ['Name', 'Type', 'Legendary']

# Sub-select by combining the lists with chosen features
df_selected = pokemon_df[number_cols + non_number_cols]

# Prints the first 5 lines of the new dataframe
print(df_selected.head())


$$$$$ Removing features without variance 3

# Leave this list as is
number_cols = ['HP', 'Attack', 'Defense']

# Remove the feature without variance from this list
non_number_cols = ['Name', 'Type']

# Create a new dataframe by subselecting the chosen features
df_selected = pokemon_df[number_cols + non_number_cols]

# Prints the first 5 lines of the new dataframe
print(df_selected.head())


$$$$$ Visually detecting redundant features 1
# Create a pairplot and color the points using the 'Gender' feature
sns.pairplot(ansur_df_1, hue='Gender', diag_kind='hist')

# Show the plot
plt.show()

$$$$$ Visually detecting redundant features 1
# Remove one of the redundant features
reduced_df = ansur_df_1.drop('body_height', axis = 1)

# Create a pairplot and color the points using the 'Gender' feature
sns.pairplot(reduced_df, hue='Gender')

# Show the plot
plt.show()

$$$$$ Visually detecting redundant features 1
# Create a pairplot and color the points using the 'Gender' feature
sns.pairplot(ansur_df_2, hue='Gender', diag_kind='hist')


# Show the plot
plt.show()

$$$$$ Visually detecting redundant features 1
# Remove the redundant feature
reduced_df = ansur_df_2.drop('n_legs', axis = 1)

# Create a pairplot and color the points using the 'Gender' feature
sns.pairplot(reduced_df, hue='Gender', diag_kind='hist')

# Show the plot
plt.show()


$$$$$ Advantage of feature selection

The selected features remain unchanged, and are therefore easy to interpret.


$$$$$ t-SNE intuition
When you want to visually explore the patterns in a high dimensional dataset


$$$$$ Fitting t-SNE to the ANSUR data
# Non-numerical columns in the dataset
non_numeric = ['Branch', 'Gender', 'Component']

# Drop the non-numerical columns from df
df_numeric = df.drop(non_numeric, axis=1)

# Create a t-SNE model with learning rate 50
m = TSNE(learning_rate = 50)

# Fit and transform the t-SNE model on the numeric dataset
tsne_features = m.fit_transform(df_numeric)
print(tsne_features)


$$$$$ t-SNE visualisation of dimensionality 1
# Color the points according to Army Component
sns.scatterplot(x="x", y="y", hue='Component', data=df)

# Show the plot
plt.show()


$$$$$ t-SNE visualisation of dimensionality 2
# Color the points by Army Branch
sns.scatterplot(x="x", y="y", hue='Branch', data=df)

# Show the plot
plt.show()


$$$$$ t-SNE visualisation of dimensionality 3
# Color the points by Gender
sns.scatterplot(x="x", y="y", hue='Gender', data=df)

# Show the plot
plt.show()
