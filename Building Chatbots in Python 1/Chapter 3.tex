######################## Building a virtual assistant ############################

$$$$$ SQL basics
SELECT name from hotels where price = 'mid' AND area = 'north'


$$$$$ SQL statements in Python
# Import sqlite3
import sqlite3

# Open connection to DB
conn = sqlite3.connect('hotels.db')

# Create a cursor
c = conn.cursor()

# Define area and price
area, price = "south", "hi"
t = (area, price)

# Execute the query
c.execute('SELECT * FROM hotels WHERE area=? AND price=?', t)

# Print the results
print(c.fetchall())


$$$$$ Creating queries from parameters
# Define find_hotels()
def find_hotels(params):
    # Create the base query
    query = 'SELECT * FROM hotels'
    # Add filter clauses for each of the parameters
    if len(params) > 0:
        filters = ["{}=?".format(k) for k in params]
        query += " WHERE " + " and ".join(filters)
    # Create the tuple of values
    t = tuple(params.values())
    
    # Open connection to DB
    conn = sqlite3.connect('hotels.db')
    # Create a cursor
    c = conn.cursor()
    # Execute the query
    c.execute(query, t)
    # Return the results
    return c.fetchall()


$$$$$ Using your custom function to find hotels
# Create the dictionary of column names and values
params = {"area" : "south", "price" : "lo"}

# Find the hotels that match the parameters
print(find_hotels(params))


$$$$$ Creating SQL from natural language
# Define respond()
def respond(message):
    # Extract the entities
    entities = interpreter.parse(message)["entities"]
    # Initialize an empty params dictionary
    params = {}
    # Fill the dictionary with entities
    for ent in entities:
        params[ent["entity"]] = str(ent["value"])

    # Find hotels that match the dictionary
    results = find_hotels(params)
    # Get the names of the hotels and index of the response
    names = [r[0] for r in results]
    n = min(len(results),3)
    # Select the nth element of the responses array
    return responses[n].format(*names)


$$$$$ Creating SQL from natural language
# Define respond()
def respond(message):
    # Extract the entities
    entities = interpreter.parse(message)["entities"]
    # Initialize an empty params dictionary
    params = {}
    # Fill the dictionary with entities
    for ent in entities:
        params[ent["entity"]] = str(ent["value"])

    # Find hotels that match the dictionary
    results = find_hotels(params)
    # Get the names of the hotels and index of the response
    names = [r[0] for r in results]
    n = min(len(results),3)
    # Select the nth element of the responses array
    return responses[n].format(*names)

# Test the respond() function
respond("I want an expensive hotel in the south of town")


$$$$$ Refining your search
# Define a respond function, taking the message and existing params as input
def respond(message, params):
    # Extract the entities
    entities = interpreter.parse(message)["entities"]
    # Fill the dictionary with entities
    for ent in entities:
        params[ent["entity"]] = str(ent["value"])

    # Find the hotels
    results = find_hotels(params)
    names = [r[0] for r in results]
    n = min(len(results), 3)
    # Return the appropriate response
    return responses[n].format(*names), params

# Initialize params dictionary
params = {}

# Pass the messages to the bot
for message in ["I want an expensive hotel", "in the north of town"]:
    print("USER: {}".format(message))
    response, params = respond(message, params)
    print("BOT: {}".format(response))


$$$$$ Basic negation
# Define negated_ents()
def negated_ents(phrase):
    # Extract the entities using keyword matching
    ents = [e for e in ["south", "north"] if e in phrase]
    # Find the index of the final character of each entity
    ends = sorted([phrase.index(e) + len(e) for e in ents])
    # Initialise a list to store sentence chunks
    chunks = []
    # Take slices of the sentence up to and including each entitiy
    start = 0
    for end in ends:
        chunks.append(phrase[start:end])
        start = end
    result = {}
    # Iterate over the chunks and look for entities
    for chunk in chunks:
        for ent in ents:
            if ent in chunk:
                # If the entity contains a negation, assign the key to be False
                if "not" in chunk or "n't" in chunk:
                    result[ent] = False
                else:
                    result[ent] = True
    return result  

# Check that the entities are correctly assigned as True or False
for test in tests:
    print(negated_ents(test[0]) == test[1])


$$$$$ Filtering with excluded slots
# Define the respond function
def respond(message, params, neg_params):
    # Extract the entities
    entities = interpreter.parse(message)["entities"]
    ent_vals = [e["value"] for e in entities]
    # Look for negated entities
    negated = negated_ents(message, ent_vals)
    for ent in entities:
        if ent["value"] in negated and negated[ent["value"]]:
            neg_params[ent["entity"]] = str(ent["value"])
        else:
            params[ent["entity"]] = str(ent["value"])
    # Find the hotels
    results = find_hotels(params,neg_params)
    names = [r[0] for r in results]
    n = min(len(results),3)
    # Return the correct response
    return responses[n].format(*names), params, neg_params

# Initialize params and neg_params
params = {}
neg_params = {}

# Pass the messages to the bot
for message in ["I want a cheap hotel", "but not in the north of town"]:
    print("USER: {}".format(message))
    response, params, neg_params = respond(message, params, neg_params)
    print("BOT: {}".format(response))


