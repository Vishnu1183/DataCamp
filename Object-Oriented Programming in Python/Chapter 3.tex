############################ Fancy classes, fancy objects #############################

##### Return Statement I
[1, 2, 3, 4, 5]


##### Return Statement II: The Return of the DataShell
# Create class: DataShell
class DataShell:
  
	# Initialize class with self and dataList as arguments
    def __init__(self, dataList):
      	# Set data as instance variable, and assign it the value of dataList
        self.data = dataList
        
	# Define method that returns data: show
    def show(self):
        return self.data
        
    # Define method that prints average of data: avg 
    def avg(self):
        # Declare avg and assign it the average of data
        avg = sum(self.data)/float(len(self.data))
        # Return avg
        return avg
        
# Instantiate DataShell taking integer_list as argument: my_data_shell
my_data_shell = DataShell(integer_list)

# Print output of your object's show method
print(my_data_shell.show())

# Print output of your object's avg method
print(my_data_shell.avg())


##### Return Statement III: A More Powerful DataShell
# Load numpy as np and pandas as pd
import numpy as np
import pandas as pd

# Create class: DataShell
class DataShell:
  
    # Initialize class with self and inputFile
    def __init__(self, inputFile):
        self.file = inputFile
        
    # Define generate_csv method, with self argument
    def generate_csv(self):
        self.data_as_csv = pd.read_csv(self.file)
        return self.data_as_csv

# Instantiate DataShell with us_life_expectancy as input argument
data_shell = DataShell(us_life_expectancy)

# Call data_shell's generate_csv method, assign it to df
df = data_shell.generate_csv()

# Print df
print(df)


##### Data as Attributes
# Import numpy as np, pandas as pd
import numpy as np
import pandas as pd

# Create class: DataShell
class DataShell:
  
    # Define initialization method
    def __init__(self, filepath):
        # Set filepath as instance variable
        self.filepath = filepath
        # Set data_as_csv as instance variable
        self.data_as_csv = pd.read_csv(filepath)

# Instantiate DataShell as us_data_shell
us_data_shell = DataShell(us_life_expectancy)

# Print your object's data_as_csv attribute
print(us_data_shell.data_as_csv)


##### Renaming Columns
# Create class DataShell
class DataShell:
  
    # Define initialization method
    def __init__(self, filepath):
        self.filepath = filepath
        self.data_as_csv = pd.read_csv(filepath)
    
    # Define method rename_column, with arguments self, column_name, and new_column_name
    def rename_column(self, column_name, new_column_name):
        self.data_as_csv.columns = self.data_as_csv.columns.str.replace(column_name, new_column_name)

# Instantiate DataShell as us_data_shell with argument us_life_expectancy
us_data_shell = DataShell(us_life_expectancy)

# Print the datatype of your object's data_as_csv attribute
print(us_data_shell.data_as_csv.dtypes)

# Rename your objects column 'code' to 'country_code'
us_data_shell.rename_column('code', 'country_code')

# Again, print the datatype of your object's data_as_csv attribute
print(us_data_shell.data_as_csv.dtypes)


##### Self-Describing DataShells
# Create class DataShell
class DataShell:

    # Define initialization method
    def __init__(self, filepath):
        self.filepath = filepath
        self.data_as_csv = pd.read_csv(filepath)

    # Define method rename_column, with arguments self, column_name, and new_column_name
    def rename_column(self, column_name, new_column_name):
        self.data_as_csv.columns = self.data_as_csv.columns.str.replace(column_name, new_column_name)
        
    # Define get_stats method, with argument self
    def get_stats(self):
        # Return a description data_as_csv
        return self.data_as_csv.describe()
    
# Instantiate DataShell as us_data_shell
us_data_shell = DataShell(us_life_expectancy)

# Print the output of your objects get_stats method
print(us_data_shell.get_stats())


##### Naming classes
CamelCase


##### Got Characters?
79


##### Which class has appropriate docstrings?
Option B

