###############  Getting ready for object-oriented programming   ######################

##### Understanding Object Dependencies
Python object.

##### Creating functions
# Create function that returns the average of an integer list
def average_numbers(num_list): 
    avg = sum(num_list)/float(len(num_list)) # divide by length of list
    return avg

# Take the average of a list: my_avg
my_avg = average_numbers([1, 2, 3, 4, 5, 6])

# Print out my_avg
print(my_avg)


##### Creating a complex data type
# Create a list that contains two lists: matrix
matrix = [[1,2,3,4] , [5,6,7,8]]

# Print the matrix list
print(matrix)


##### What are NumPy Arrays most similar to?
Lists


##### Create a function that returns a NumPy array
# Import numpy as np
import numpy  as np

# List input: my_matrix
my_matrix = [[1,2,3,4], [5,6,7,8]] 

# Function that converts lists to arrays: return_array
def return_array(matrix):
    array = np.array(matrix, dtype = float)
    return array
    
# Call return_array on my_matrix, and print the output
print(return_array(my_matrix))


##### Creating a class
# Create a class: DataShell
class DataShell: 
    pass


##### Difference between a class and an object
Objects, classes.
