####################### Can I Run My Application in Parallel?   #######################

##### Partitioning demographic model
Submodels Births, Deaths and Migration can run in parallel but not Population.


##### Partitioning probabilistic demographic model
B and D


##### Find the most frequent words in a text



#####



##### A simple embarrassingly parallel application -- 1
# Complete the function definition
mean_of_rnorm <- function(n) {
  # Generate normally distributed random numbers
  random_numbers <- rnorm(n)
  # Calculate the mean of the random numbers
  return(mean(random_numbers))
}

# Try it out
mean_of_rnorm(100)


##### A simple embarrassingly parallel application -- 2
# From previous step
mean_of_rnorm <- function(n) {
  random_numbers <- rnorm(n)
  mean(random_numbers)
}

# Create a vector to store the results
result <- rep(NA,n_replicates)

# Set the random seed to 123
set.seed(123)

# Set up a for loop with iter from 1 to n_replicates
for (iter in 1:n_replicates) {
  # Call mean_of_rnorm with n_numbers_per_replicate
  result[iter] <- mean_of_rnorm(n_numbers_per_replicate)
}

# View the result
hist(result)



##### A simple embarrassingly parallel application -- 3
# From previous step
mean_of_rnorm <- function(n) {
  random_numbers <- rnorm(n)
  mean(random_numbers)
}

# Repeat n_numbers_per_replicate, n_replicates times
n <- rep(n_numbers_per_replicate, n_replicates)

# Call mean_of_rnorm() repeatedly using sapply()
result <- sapply(
  # The vectorized argument to pass
  n, 
  # The function to call
  mean_of_rnorm
)

# View the results
hist(result)


##### Probabilistic projection of migration (setup)
Matrix where traj_len corresponds to columns and block_size corresponds to rows.

##### Probabilistic projection of migration
# Function definition of ar1_multiple_blocks_of_trajectories()
ar1_multiple_blocks_of_trajectories <- function(ids, ...) {
  # Call ar1_block_of_trajectories() for each ids
  trajectories_by_block <- lapply(ids, ar1_block_of_trajectories, ...)
  
  # rbind results
  do.call(rbind, trajectories_by_block)
}

# Create a sequence from 1 to number of blocks
traj_ids <- seq_len(nrow(ar1est))

# Generate trajectories for all rows of the estimation dataset
trajs <- ar1_multiple_blocks_of_trajectories(
  ids = traj_ids, rate0 = 0.015,
  block_size = 10, traj_len = 15
)

# Show results
show_migration(trajs)


##### Passing arguments via clusterApply()  -- 1
# Load parallel
library(parallel)

# How many physical cores are available?
ncores = detectCores(logical =FALSE)

# How many random numbers to generate
n = seq(ncores:1)



##### Passing arguments via clusterApply()  -- 2
# From previous step
library(parallel)
ncores <- detectCores(logical = FALSE)
n <- ncores:1

# Use lapply to call rnorm for each n,
# setting mean to 10 and sd to 2 
lapply(n, rnorm, mean = 10, sd = 2)



##### Passing arguments via clusterApply()  -- 3
# From previous step
library(parallel)
ncores <- detectCores(logical = FALSE)
n <- ncores:1

# Create a cluster
cl <- makeCluster(ncores)

# Use clusterApply to call rnorm for each n in parallel,
# again setting mean to 10 and sd to 2 
clusterApply(cl, n, rnorm, mean = 10, sd = 2)

# Stop the cluster
stopCluster(cl)


#####  Sum in parallel
# Evaluate partial sums in parallel
part_sums <- clusterApply(cl, x = c(1, 51),
                    fun = function(x) sum(x:(x + 49)))
# Total sum
total <- sum(unlist(part_sums))

# Check for correctness
total == sum(1:100)


##### More tasks than workers
# Create a cluster and set parameters
cl <- makeCluster(2)
n_replicates <- 50
n_numbers_per_replicate <- 10000

# Repeat n_numbers_per_replicate, n_replicates times
n <- rep(n_numbers_per_replicate, n_replicates)

# Parallel evaluation
means <- clusterApply(cl, x = n, fun = mean_of_rnorm)
                
# View results as histogram
hist(unlist(means))
