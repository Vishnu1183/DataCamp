########################    Random Numbers and Reproducibility   ###############################


##### Reproducibility (I)
No, because it sets the RNG seed only on the master process.


##### SOCK vs. FORK  --1
# Register the SOCK cluster
cl.sock <- makeCluster(2, type = "SOCK")
registerDoParallel(cl.sock)
replicate(
  # Use 2 replicates
  n = 2, 
  expr = {
    # Set the seed to 100
    set.seed(100)
    # Run two iterations in parallel, bound by rows
    foreach(i = 1:2, .combine = rbind) %dopar% rnorm(3)
  }, 
  simplify = FALSE
)



##### SOCK vs. FORK  --2
# Change this to register the FORK cluster
registerDoParallel(cl.fork)

# Run this again and look at the output!
replicate(
  n = 2, 
  expr = {
    set.seed(100)
    foreach(i = 1:2, .combine = rbind) %dopar% rnorm(3)
  }, 
  simplify = FALSE
)



##### SOCK vs. FORK & random numbers
FORK generates the same random numbers on each node because nodes are copies of the master.


##### Setting an RNG
# Create a cluster
cl <- makeCluster(2)

# Check RNGkind on workers
clusterCall(cl, RNGkind)

# Set the RNG seed on workers
clusterSetRNGStream(cl, 100)

# Check RNGkind on workers
clusterCall(cl, RNGkind)





##### Reproducible results in parallel  --1
# The cluster, & how many numbers to generate
cl
n_vec

t(replicate(
  # Use 3 replicates
  n = 3,
  expr = {
    # Spread across cl, apply mean_of_rnorm() to n_vec
    clusterApply(cl, n_vec,mean_of_rnorm)
  }
))





##### Reproducible results in parallel  --2
t(replicate(
  n = 3,
  expr = {
    # Set the cluster's RNG stream seed to 1234
    clusterSetRNGStream(cl, iseed = 1234)
    clusterApply(cl, n_vec, mean_of_rnorm)
  }
))





##### Non-reproducible results in parallel --1
# Make a cluster of size 2
cl2 <- makeCluster(2)

# Set the cluster's RNG stream seed to 1234
clusterSetRNGStream(cl2, iseed = 1234)

# Spread across the cluster, apply mean_of_rnorm() to n_vec
unlist(clusterApply(cl2, n_vec, mean_of_rnorm))




##### Non-reproducible results in parallel --2
# From previous step
cl2 <- makeCluster(2)
clusterSetRNGStream(cl2, iseed = seed)
unlist(clusterApply(cl2, n_vec, mean_of_rnorm))

# Do the same, with 4 nodes
cl4 <- makeCluster(4)
clusterSetRNGStream(cl4, iseed = seed)
unlist(clusterApply(cl4, n_vec, mean_of_rnorm))



##### Reproducibility (II) 
Yes. It registers the L'Ecuyer RNG, sets its seed and each replication uses its own stream.


##### Reproducing migration app with foreach
# Register doParallel and doRNG
registerDoParallel(cores = 2)
registerDoRNG(seed)

# Run foreach
mpar <- foreach(r = 1:5) %dopar% ar1_block_of_trajectories(r)

# Register sequential backend, set seed and repeat foreach
registerDoSEQ()
set.seed(seed)
mseq <- foreach(r = 1:5) %dorng% ar1_block_of_trajectories(r)

# Check if results identical
identical(mpar, mseq)





##### Reproducing migration app with future.apply
# Set multiprocess plan 
plan(multiprocess, workers = 2)

# Call ar1_block_of_trajectories via future_lapply
mfpar <- future_lapply(1:5, ar1_block_of_trajectories, future.seed = seed)

# Repeat for sequential plan
plan(sequential)
mfseq <- future_lapply(1:5, ar1_block_of_trajectories, future.seed = seed)

# Check if results are identical
identical(mfpar, mfseq)


