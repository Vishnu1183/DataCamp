############################## Fine Tuning: Efficient Base R   ########################

##### Why is this piece of code slow?
We're committing the cardinal sin - growing the vector.

##### Timings - growing a vector
# Use <- with system.time() to store the result as res_grow
system.time(res_grow <- growing(n=30000))

##### Timings - pre-allocation
# Use <- with system.time() to store the result as res_allocate
n <- 30000
system.time(res_allocate <- pre_allocate(n))

##### Vectorized code: multiplication
# Store your answer as x2_imp
x2_imp <- x**2

##### Vectorized code: calculating a log-sum
# Initial code
n <- 100
total <- 0
x <- runif(n)
for(i in 1:n) 
    total <- total + log(x[i])

# Rewrite in a single line. Store the result in log_sum
log_sum <- sum(log(x))

##### Data frames vs matrices
All values in a column of a data frame must have the same data type.

##### Data frames and matrices - column selection
# Which is faster, mat[, 1] or df[, 1]? 
microbenchmark(mat[,1], df[,1])

##### Selecting a row in a data frame
A data frame

##### Row timings
# Which is faster, mat[1, ] or df[1, ]? 
microbenchmark( mat[1, ] , df[1, ])
