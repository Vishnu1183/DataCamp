############################## Diagnosing Problems: Code Profiling  ######################

##### Profiling a function
It records the call stack at regular intervals.

##### Where's the hold-up?
Line 7: Calculating the loess smoother

##### Profvis in action
# Load the data set
data(movies, package = "ggplot2movies") 

# Load the profvis package
library(profvis)

# Profile the following code with the profvis function
profvis({
  # Load and select data
  movies <- movies[movies$Comedy == 1, ]

  # Plot data of interest
  plot(movies$year, movies$rating)

  # Loess regression line
  model <- loess(rating ~ year, data = movies)
  j <- order(movies$year)
  
  # Add a fitted line to the plot
  lines(movies$year[j], model$fitted[j], col = "red")
} )    ## Remember the closing brackets!

##### Change the data frame to a matrix
# Load the microbenchmark package
library(microbenchmark)

# The previous data frame solution is defined
# d() Simulates 6 dices rolls
d <- function() {
  data.frame(
    d1 = sample(1:6, 3, replace = TRUE),
    d2 = sample(1:6, 3, replace = TRUE)
  )
}

# Complete the matrix solution
m <- function() {
  matrix(sample(1:6, 6, replace = TRUE), ncol = 2)
}

# Use microbenchmark to time m() and d()
microbenchmark(
 data.frame_solution = d(),
 matrix_solution     = m()
)

##### Calculating row sums
# Example data
rolls

# Define the previous solution 
app <- function(x) {
    apply(x, 1, sum)
}

# Define the new solution
r_sum <- function(x) {
    rowSums(x)
}

# Compare the methods
microbenchmark(
    app_sol = app(rolls),
    r_sum_sol = r_sum(rolls)
)

##### Use && instead of &
# Example data
is_double

# Define the previous solution
move <- function(is_double) {
    if (is_double[1] & is_double[2] & is_double[3]) {
        current <- 11 # Go To Jail
    }
}

# Define the improved solution
improved_move <- function(is_double) {
    if (is_double[1] && is_double[2] && is_double[3]) {
        current <- 11 # Go To Jail
    }
}

# microbenchmark both solutions
# Very occassionally the improved solution is actually a little slower
# This is just random chance
microbenchmark(move, improved_move, times = 1e5)
