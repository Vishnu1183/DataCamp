######################  Support Vector Classifiers - Linear Kernels ##############################

$$$$$ Creating training and test datasets
#split train and test data in an 80/20 proportion
df[, "train"] <- ifelse(runif(nrow(df))<0.8, 1, 0)

#assign training rows to data frame trainset
trainset <- df[df$train == 1, ]
#assign test rows to data frame testset
testset <- df[df$train == 0, ]

#find index of "train" column
trainColNum <- grep("train", names(df))

#remove "train" column from train and test dataset
trainset <- trainset[, -trainColNum]
testset <- testset[, -trainColNum]


$$$$$ Building a linear SVM classifier
library(e1071)

#build svm model, setting required parameters
svm_model<- svm(y ~ ., 
                data = trainset, 
                type = "C-classification", 
                kernel = "linear", 
                scale = FALSE)


$$$$$ Exploring the model and calculating accuracy  
#list components of model
names(svm_model)


$$$$$ Exploring the model and calculating accuracy
#list components of model
names(svm_model)

#list values of the SV, index and rho
svm_model$SV
svm_model$index
svm_model$rho


$$$$$ Exploring the model and calculating accuracy
#list components of model
names(svm_model)

#list values of the SV, index and rho
svm_model$SV
svm_model$index
svm_model$rho

#compute training accuracy
pred_train <- predict(svm_model, trainset)
mean(pred_train == trainset$y)


$$$$$ Exploring the model and calculating accuracy
#list components of model
names(svm_model)

#list values of the SV, index and rho
svm_model$SV
svm_model$index
svm_model$rho

#compute training accuracy
pred_train <- predict(svm_model, trainset)
mean(pred_train == trainset$y)

#compute test accuracy
pred_test <- predict(svm_model, testset)
mean(pred_test == testset$y)


$$$$$ Visualizing support vectors using ggplot
#load ggplot
library(ggplot2)

#build scatter plot of training dataset
scatter_plot <- ggplot(data = trainset, aes(x = x1, y = x2, color = y)) + 
    geom_point() + 
    scale_color_manual(values = c("red", "blue"))
 
#add plot layer marking out the support vectors 
layered_plot <- 
    scatter_plot + geom_point(data = trainset[svm_model$index, ], aes(x = x1, y = x2), color = "purple", size = 4, alpha = 0.5)

#display plot
layered_plot


$$$$$ Visualizing decision & margin bounds using `ggplot2`
#calculate slope and intercept of decision boundary from weight vector and svm model
slope_1 <- -w[1]/w[2]
intercept_1 <- svm_model$rho/w[2]

#build scatter plot of training dataset
scatter_plot <- ggplot(data = trainset, aes(x = x1, y = x2, color = y)) + 
    geom_point() + scale_color_manual(values = c("red", "blue"))
#add decision boundary 
plot_decision <- scatter_plot + geom_abline(slope = slope_1, intercept = intercept_1) 
#add margin boundaries
plot_margins <- plot_decision + 
 geom_abline(slope = slope_1, intercept = intercept_1 - 1/w[2], linetype = "dashed")+
 geom_abline(slope = slope_1, intercept = intercept_1 + 1/w[2], linetype = "dashed")
#display plot
plot_margins


$$$$$ Visualizing decision & margin bounds using `plot()`
#load required library
library(e1071)

#build svm model
svm_model<- 
    svm(y ~ ., data = trainset, type = "C-classification", 
        kernel = "linear", scale = FALSE)

#plot decision boundaries and support vectors for the training data
plot(x = svm_model, data = trainset)


$$$$$ Tuning a linear SVM
#build svm model, cost = 1
svm_model_1 <- svm(y ~ .,
                   data = trainset,
                   type = "C-classification",
                   cost = 1,
                   kernel = "linear",
                   scale = FALSE)

#print model details
summary(svm_model_1)


$$$$$ Tuning a linear SVM
#build svm model, cost = 100
svm_model_100 <- svm(y ~ .,
                   data = trainset,
                   type = "C-classification",
                   cost = 100,
                   kernel = "linear",
                   scale = FALSE)

#print model details
summary(svm_model_100)


$$$$$ Visualizing decision boundaries and margins
#add decision boundary and margins for cost = 1 to training data scatter plot
train_plot_with_margins <- train_plot + 
    geom_abline(slope = slope_1, intercept = intercept_1) +
    geom_abline(slope = slope_1, intercept = intercept_1-1/w_1[2], linetype = "dashed")+
    geom_abline(slope = slope_1, intercept = intercept_1+1/w_1[2], linetype = "dashed")

#display plot
train_plot_with_margins


$$$$$ Visualizing decision boundaries and margins
#add decision boundary and margins for cost = 100 to training data scatter plot
train_plot_with_margins <- train_plot_100 + 
    geom_abline(slope = slope_100, intercept = intercept_100, color = "goldenrod") +
    geom_abline(slope = slope_100, intercept = intercept_100-1/w_100[2], linetype = "dashed", color = "goldenrod")+
    geom_abline(slope = slope_100, intercept = intercept_100+1/w_100[2], linetype = "dashed", color = "goldenrod")

#display plot 
train_plot_with_margins


$$$$$ When are soft margin classifiers useful?
Working with a dataset that is almost linearly separable.


$$$$$ A multiclass classification problem
#load library and build svm model
library(e1071)
svm_model<- 
    svm(y ~ ., data = trainset, type = "C-classification", 
        kernel = 'linear', scale = FALSE)


$$$$$ A multiclass classification problem
#load library and build svm model
library(e1071)
svm_model<- 
    svm(y ~ ., data = trainset, type = "C-classification", 
        kernel = "linear", scale = FALSE)

#compute training accuracy
pred_train <- predict(svm_model, trainset)
mean(pred_train == trainset$y)



$$$$$ A multiclass classification problem
#load library and build svm model
library(e1071)
svm_model<- 
    svm(y ~ ., data = trainset, type = "C-classification", 
        kernel = "linear", scale = FALSE)

#compute training accuracy
pred_train <- predict(svm_model, trainset)
mean(pred_train == trainset$y)

#compute test accuracy
pred_test <- predict(svm_model, testset)
mean(pred_test == testset$y)



$$$$$ A multiclass classification problem
#load library and build svm model
library(e1071)
svm_model<- 
    svm(y ~ ., data = trainset, type = "C-classification", 
        kernel = "linear", scale = FALSE)

#compute training accuracy
pred_train <- predict(svm_model, trainset)
mean(pred_train == trainset$y)

#compute test accuracy
pred_test <- predict(svm_model, testset)
mean(pred_test == testset$y)

#plot
plot(svm_model, trainset)



$$$$$ Iris redux - a more robust accuracy.
for (i in 1:100){
  	#assign 80% of the data to the training set
    iris[, "train"] <- ifelse(runif(nrow(iris)) < 0.8, 1, 0)
    trainColNum <- grep("train", names(iris))
    trainset <- iris[iris$train == 1, -trainColNum]
    testset <- iris[iris$train == 0, -trainColNum]
  	#build model using training data
    svm_model <- svm(Species~ ., data = trainset, 
                     type = "C-classification", kernel = "linear")
  	#calculate accuracy on test data
    pred_test <- predict(svm_model, testset)
    accuracy[i] <- mean(pred_test == testset$Species)
}
mean(accuracy) 
sd(accuracy)

