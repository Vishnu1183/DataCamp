#################################   Processing Pipelines    ############################


##### What happens when you call nlp?
Tokenize the text and apply each pipeline component in order.

##### Inspecting the pipeline
# Load the en_core_web_sm model
nlp = spacy.load('en_core_web_sm')

# Print the names of the pipeline components
print(nlp.pipe_names)

# Print the full pipeline of (name, component) tuples
print(nlp.pipeline)


##### Use cases for custom components
2 and 3


##### Simple components  --1
# Define the custom component
def length_component(doc):
    # Get the doc's length
    doc_length = len(doc)
    print("This document is {} tokens long.".format(doc_length))
    # Return the doc
    return doc



##### Simple components --2
# Define the custom component
def length_component(doc):
    # Get the doc's length
    doc_length = len(doc)
    print("This document is {} tokens long.".format(doc_length))
    # Return the doc
    return doc

# Load the small English model
nlp = spacy.load('en_core_web_sm')
  
# Add the component first in the pipeline and print the pipe names
nlp.add_pipe(length_component, first=True)
print(nlp.pipe_names)



##### Simple components --3
# Define the custom component
def length_component(doc):
    # Get the doc's length
    doc_length = len(doc)
    print("This document is {} tokens long.".format(doc_length))
    # Return the doc
    return doc
  
# Load the small English model and Add the component first in the pipeline
nlp = spacy.load('en_core_web_sm')
nlp.add_pipe(length_component, first=True)

# Process a text
doc = doc = nlp("This is a sentence.")


##### Complex components --1
# Define the custom component
def animal_component(doc):
    # Apply the matcher to the doc
    matches = matcher(doc)
    # Create a Span for each match and assign the label 'ANIMAL'
    spans = [Span(doc, start, end, label='ANIMAL')
             for match_id, start, end in matches]
    # Overwrite the doc.ents with the matched spans
    doc.ents = spans
    return doc



##### Complex components --2
# Define the custom component
def animal_component(doc):
    # Apply the matcher to the doc
    matches = matcher(doc)
    # Create a Span for each match and assign the label 'ANIMAL'
    spans = [Span(doc, start, end, label='ANIMAL')
             for match_id, start, end in matches]
    # Overwrite the doc.ents with the matched spans
    doc.ents = spans
    return doc

# Add the component to the pipeline after the 'ner' component 
nlp.add_pipe(animal_component, after = 'ner')
print(nlp.pipe_names)



#####  Complex components --3
# Define the custom component
def animal_component(doc):
    # Create a Span for each match and assign the label 'ANIMAL'
    # and overwrite the doc.ents with the matched spans
    doc.ents = [Span(doc, start, end, label='ANIMAL')
                for match_id, start, end in matcher(doc)]
    return doc
    
# Add the component to the pipeline after the 'ner' component 
nlp.add_pipe(animal_component, after='ner')

# Process the text and print the text and label for the doc.ents
doc = nlp("I have a cat and a Golden Retriever")
print([(ent.text, ent.label_) for ent in doc.ents])



##### Setting extension attributes (1)  --- 1
# Register the Token extension attribute 'is_country' with the default value False
Token.set_extension('is_country', default=False)

# Process the text and set the is_country attribute to True for the token "Spain"
doc = nlp("I live in Spain.")
doc[3]._.is_country = True

# Print the token text and the is_country attribute for all tokens
print([(token.text, token._.is_country) for token in doc])


##### Setting extension attributes (1) ---1
# Define the getter function that takes a token and returns its reversed text
def get_reversed(token):
    return token.text[::-1]
  
# Register the Token property extension 'reversed' with the getter get_reversed
Token.set_extension('reversed', getter=get_reversed)

# Process the text and print the reversed attribute for each token
doc = nlp("All generalizations are false, including this one.")
for token in doc:
    print('reversed:',  token._.reversed)



##### Setting extension attributes (2)  ---1
# Define the getter function
def get_has_number(doc):
    # Return if any of the tokens in the doc return True for token.like_num
    return any(token.like_num for token in doc)

# Register the Doc property extension 'has_number' with the getter get_has_number
Doc.set_extension('has_number', getter=get_has_number)

# Process the text and check the custom has_number attribute 
doc = nlp("The museum closed for five years in 2012.")
print('has_number:', doc._.has_number)



##### Setting extension attributes (2)  ---2
# Define the method
def to_html(span, tag):
    # Wrap the span text in a HTML tag and return it
    return '<{tag}>{text}</{tag}>'.format(tag=tag, text=span.text)

# Register the Span property extension 'to_html' with the method to_html
Span.set_extension('to_html', method=to_html)

# Process the text and call the to_html method on the span with the tag name 'strong'
doc = nlp("Hello world, this is a sentence.")
span = doc[0:2]
print(span._.to_html('strong'))


##### Entities and extensions
def get_wikipedia_url(span):
    # Get a Wikipedia URL if the span has one of the labels
    if span.label_ in ('PERSON', 'ORG', 'GPE', 'LOCATION'):
        entity_text = span.text.replace(' ', '_')
        return "https://en.wikipedia.org/w/index.php?search=" + entity_text

# Set the Span extension wikipedia_url using get getter get_wikipedia_url
Span.set_extension('wikipedia_url', getter=get_wikipedia_url)

doc = nlp("In over fifty years from his very first recordings right through to his last album, David Bowie was at the vanguard of contemporary culture.")
for ent in doc.ents:
    # Print the text and Wikipedia URL of the entity
    print(ent.text, ent._.wikipedia_url)


##### Components with extensions  --1
def countries_component(doc):
    # Create an entity Span with the label 'GPE' for all matches
    doc.ents = [Span(doc , start, end, label='GPE')
                for match_id, start, end in matcher(doc)]
    return doc

# Add the component to the pipeline
nlp.add_pipe(countries_component)
print(nlp.pipe_names)

##### Components with extensions --2
def countries_component(doc):
    # Create an entity Span with the label 'GPE' for all matches
    doc.ents = [Span(doc, start, end, label='GPE')
                for match_id, start, end in matcher(doc)]
    return doc

# Add the component to the pipeline
nlp.add_pipe(countries_component)

# Getter that looks up the span text in the dictionary of country capitals
get_capital = lambda span: capitals.get(span.text)

# Register the Span extension attribute 'capital' with the getter get_capital 
Span.set_extension('capital', getter = get_capital)

##### Components with extensions --3
def countries_component(doc):
    # Create an entity Span with the label 'GPE' for all matches
    doc.ents = [Span(doc, start, end, label='GPE')
                for match_id, start, end in matcher(doc)]
    return doc

# Add the component to the pipeline
nlp.add_pipe(countries_component)

# Register capital and getter that looks up the span text in country capitals
Span.set_extension('capital', getter=lambda span: capitals.get(span.text))

# Process the text and print the entity text, label and capital attributes
doc = nlp("Czech Republic may help Slovakia protect its airspace")
print([(ent.text, ent.label_, ent._.capital) for ent in doc.ents])

##### Processing streams --- 1
# Process the texts and print the adjectives
for doc in nlp.pipe(TEXTS):
    print([token.text for token in doc if token.pos_ == 'ADJ'])


##### Processing streams --- 2
# Process the texts and print the entities
docs = list(nlp.pipe(TEXTS))
entities = [doc.ents for doc in docs]
print(*entities)

##### Processing streams --- 3
people = ['David Bowie', 'Angela Merkel', 'Lady Gaga']

# Create a list of patterns for the PhraseMatcher
patterns = list(nlp.pipe(people))
