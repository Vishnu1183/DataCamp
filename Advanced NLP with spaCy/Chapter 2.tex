####################   Large-scale data analysis with spaCy   ########################


##### Strings to hashes  --1
# Look up the hash for the word "cat"
cat_hash = nlp.vocab.strings['cat']
print(cat_hash)

# Look up the cat_hash to get the string
cat_string = nlp.vocab.strings[cat_hash]
print(cat_string)



##### Strings to hashes --2
# Look up the hash for the string label "PERSON"
person_hash = nlp.vocab.strings["PERSON"]
print(person_hash)

# Look up the person_hash to get the string
person_string = nlp.vocab.strings[person_hash]
print(person_string)


##### Vocab, hashes and lexemes
The string 'Bowie' isn't present in the German vocab, so the hash can't be resolved in the string store.


##### Creating a Doc    -- 1
# Import the Doc class
from spacy.tokens import Doc

# Desired text: "spaCy is cool!"
words = ['spaCy', 'is', 'cool', '!']
spaces = [True, True, False, False]

# Create a Doc from the words and spaces
doc = Doc(nlp.vocab, words=words, spaces=spaces)
print(doc.text)



##### Creating a Doc    -- 2
# Import the Doc class
from spacy.tokens import Doc

# Desired text: "Go, get started!"
words = ['Go', ',', 'get', 'started', '!']
spaces = [False, True, True, False, False]

# Create a Doc from the words and spaces
doc = Doc(nlp.vocab, words=words, spaces=spaces)
print(doc.text)



##### Creating a Doc    -- 3
# Import the Doc class
from spacy.tokens import Doc

# Desired text: "Oh, really?!"
words = ['Oh',',', 'really','?', '!']
spaces = [False, True, False, False, False]

# Create a Doc from the words and spaces
doc = Doc(nlp.vocab, words=words, spaces=spaces)
print(doc.text)



##### Docs, spans and entities from scratch   --1
# Import the Doc and Span classes
from spacy.tokens import Doc, Span

words = ['I', 'like', 'David', 'Bowie']
spaces = [True, True, True, False]

# Create a doc from the words and spaces
doc = Doc(nlp.vocab, words=words, spaces=spaces)
print(doc.text)



##### Docs, spans and entities from scratch  --2
# Import the Doc and Span classes
from spacy.tokens import Doc, Span

# Create a doc from the words and spaces
doc = Doc(nlp.vocab, words=['I', 'like', 'David', 'Bowie'], spaces=[True, True, True, False])

# Create a span for "David Bowie" from the doc and assign it the label "PERSON"
span = Span(doc, 2, 4, label="PERSON")
print(span.text, span.label_)



##### Docs, spans and entities from scratch  --3
# Import the Doc and Span classes
from spacy.tokens import Doc, Span

# Create a doc from the words and spaces
doc = Doc(nlp.vocab, words=['I', 'like', 'David', 'Bowie'], spaces=[True, True, True, False])

# Create a span for "David Bowie" from the doc and assign it the label "PERSON"
span = Span(doc, 2, 4, label='PERSON')

# Add the span to the doc's entities
doc.ents = [span]

# Print entities' text and labels
print([(ent.text, ent.label_) for ent in doc.ents])



##### Data structures best practices  --1
It only uses lists of strings instead of native token attributes. This is often less efficient, and can't express complex relationships.

##### Data structures best practices  --2
# Iterate over the tokens
for token in doc:
    # Check if the current token is a proper noun
    if token.pos_ == 'PROPN':
        # Check if the next token is a verb
        if doc[token.i + 1].pos_ == 'VERB':
            print('Found a verb after a proper noun!')



